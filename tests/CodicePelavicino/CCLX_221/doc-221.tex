

				\documentclass[a4paper]{article}
				\usepackage[T1]{fontenc}
				\usepackage[utf8x]{inputenc}
				\usepackage[latin,italian]{babel}
				\usepackage{xcolor}  
				\usepackage{footmisc}
				\usepackage{soulutf8}
				\usepackage{amsmath}


				\definecolor{ashgrey}{rgb}{0.7, 0.75, 0.71}

				\newcommand{\evidenzia}[2]{\sethlcolor{#1}\texthl{#2}}
				\newcommand{\persona}[1]{\evidenzia{yellow}{#1}}
				\newcommand{\luogo}[1]{\evidenzia{green}{#1}}
				\newcommand{\professione}[1]{\evidenzia{magenta}{#1}}
				\newcommand{\data}[1]{\evidenzia{ashgrey}{#1}}
				\newcommand{\luoghiinteresse}[1]{\evidenzia{red}{#1}}
				\newcommand{\monete}[1]{\evidenzia{cyan}{#1}}

			
		\begin{document} 
	\title{221 (CCLX)}\date{}\maketitle
        1268 dicembre 15, Sarzanello (in castro Sarzane, in palacio domini
          episcopi)
      \\\par Rollando, figlio del fu Enrico Blanco di Erberia, dona a Guglielmo, vescovo di
          Luni, beni e diritti a lui spettanti in Soleria, Moncigolo e Ceserano. La
          donazione è fatta per la rimessione dei peccati e per i danni inferti al vescovo
          quando Rollando era insieme a Bernabò Malaspina, nemico e persecutore della
          chiesa di Luni, e a Manfredi, principe di Taranto.\\\par 
          Copia [C] CP cc. 267v.- 268r., cc. CCXIIIIv.- CCLXr. secondo la numerazione
            originale.
        \par 
          Edizione parziale in LUPO GENTILE
            1912, n. 260, pp. 239-240
        Nel margine esterno, di mano moderna: Soleria, Monciculo e Ciserano. \\\par  [1] In nomine Domini, amen.\data{ Anno a
            nativitate Eius millesimo CCLXVIII, indictione XI, XV die decembris}.
          Dominus \persona{Rollandus quondam domini
            Henrici Blanci de }\luogo{Herberia}\persona{},
          per se suosque heredes, donavit inrevocabiliter inter vivos venerabili patri
          domino \persona{Guillelmo, Dei gratia }\professione{Lunensi
              episcopo}\persona{}, recipienti pro se suisque successoribus
          et nomine episcopatus, atque dedit, tradidit, cessit et obtulit eidem domino
          episcopo, pro anima sua et in remissionem peccatorum suorum et pro iniuriis
          atque dampnis datis et illatis ipsi domino episcopo vel Lunensi episcopatui
          super eo quod ipse dominus \persona{Rollandus} adhesit olim domino \persona{Bernabo\footnote{Bernabo: \emph{così nel
            testo}.}, marchioni Malaspine}, inimico et persecutori
          tunc Lunensis ecclesie contra ipsam ecclesiam et factori \persona{Manfredi, olim principis }\luogo{Tarentini}\persona{},
          portionem suam sive partem sibi contingentem in \luogo{Soleria}, \luogo{Monciculo} et
            \luogo{Cisirano} atque districtibus et
          pertinenciis ipsorum locorum atque totum et quicquid habet vel ei competit,
          aliquo modo vel iure, in predictis locis in iurisdictione, dominio, hominibus,
          villanis seu vassallis, ambaxiariis seu ambaxiatoribus, redditibus, fictis et
          pensionibus, terris et possessionibus, agris et cultis, silvis et nemoribus,
          plenis et vacuis, divisis atque indivisis, aquis aquarumque discursibus,
          piscationibus atque venationibus atque aliis omnibus atque  quibuscumque rebus ad ipsum dominum \persona{Rollandum} quoquomodo spectantibus
          in dictis terris sive locis et villis et pertinenciis ipsorum locorum. Et dedit,
          tradidit, cessit atque mandavit idem dominus \persona{Rollandus} eidem domino episcopo,
          recipienti ut dictum est, omne dominium, proprietatem, usum et utilitatem et
          omnia iura, actiones et rationes, utiles et directas, reales, personales et
          mixtas et omnes alias sibi competentes et competentia seu competitura in
          predictis omnibus et singulis vel eorum occasione, faciendo eundem dominum
          episcopum de predictis procuratorem et dominum ut in rem suam, ita ut ex hiis
          omnibus et singulis possit agere et experiri, tueri, causari et deffendere, suo
          proprio nomine directo et utiliter, adversus quamcumque personam et locum et
          exinde facere quicquid voluerit, sine contradictione dicti domini \persona{Rollandi} vel eius heredum seu
          alterius persone. Et dedit idem dominus \persona{Rollandus} eidem domino episcopo
          licenciam, tenutam et possessionem corporalem vel quasi intrandi\footnote{intrandi: \emph{foro nella pergamena preesistente alla
              scrittura}.} de predictis, sua auctoritate, quandocumque
          voluerit, constituendo se pro eo et eius nomine possidere vel quasi donec
          tenutam et possessionem corporalem vel quasi de predictis fuerit introgressus.
          Hanc autem donationem, dationem et traditionem, datum seu cessionem et
          oblationem et omnia suprascripta et infrascripta dictus dominus \persona{Rollandus} per se suosque heredes
          promisit et convenit eidem domino episcopo, recipienti pro se suisque
          successoribus vel cui dederit vel commiserit et nomine sui episcopatus, ratam
          seu ratum habere et tenere, attendere et observare et numquam in aliquo contra
          facere vel venire, de iure vel de facto, sub pena \monete{CC librarum Ianuensium}; quam penam
          eidem domino episcopo vel suis successoribus dare et solvere promisit idem
          dominus \persona{Rollandus} tociens
          quociens fuerit contraventum et pro quolibet capitulo non observato; et, pena
          soluta vel non, predicta omnia et singula in sua permaneant firmitate, obligando
          ad hec attendenda idem dominus \persona{Rollandus} se suosque heredes et omnia sua bona mobilia, habita
          et habenda, retentis solomodo\footnote{solomodo: \emph{così nel testo}.}
          fructibus et usufructibus omnium predictorum sibi in vita sua tantum ipse
          dominus \persona{Rollandus} et
          renunciando omni iuri et legibus, quibus contra predicta vel aliquod predictorum
          posset facere vel venire, et specialiter legi que dicit donationem vel cessionem
          posse rescindi vel retractari per ingratitudinem vel alio quocumque modo. \\\par  [2] Actum in \luogo{castro
            Sarzane}, in palacio predicti domini episcopi, presentibus
            \persona{Peregrino }\professione{canonico
              Lunensi}\persona{}, \persona{Iacobino
              }\professione{archipresbytero de }\luogo{}\luoghiinteresse{Amelia}\luogo{}\professione{}\persona{} et \persona{Melanense Brixiolini, cive }\luogo{Lucano}\persona{}, testibus ad hec rogatis. \\\par  [3]  Ego
            \persona{Petrus quondam Gualteroti de }\luogo{Soleria}\persona{, }\professione{sacri palacii
              notarius}\persona{}, hiis interfui et rogatus hanc cartam
          scripsi.
       \end{document}